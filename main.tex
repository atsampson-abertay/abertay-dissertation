% Example main source file.

\documentclass{abertayhons}

\addbibresource{refs.bib}

\usepackage{hyperref}

% \blindtext is used in various places to generate some example text;
% in your real dissertation you won't need this package.
\usepackage{blindtext}

\begin{document}

\frontmatter

\title{The construction of a \LaTeX\ class for Abertay honours
dissertations}
\author{Adam Sampson}
\degree{BSc (Hons) Underwater Basket Weaving}
\year{2018}
\school{School of Design and Informatics}
\maketitle

\tablematter

\tableofcontents

\listoffigures

\listoftables

\frontmatter

\begin{acknowledgements}
\Blindtext[1]
\end{acknowledgements}

\begin{abstract}
\Blindtext[2]
\end{abstract}

\mainmatter

\chapter{Introduction}

The \LaTeX\ system is pretty good for formatting
%dissertations~\citep{wblatex}.
\blindtext

As shown by~\cite{llvm}, the LLVM compiler framework is also really cool.
\blindtext

\blindtext
The implementation of this algorithm is shown in~\autoref{fig:somecode}.

\begin{figure}
% See the "listings" package for fancier code formatting.
\begin{verbatim}
// Work out the point in the complex plane that
// corresponds to this pixel in the output image.
complex<double> c(left + (x * (right - left) / WIDTH),
		  top + (y * (bottom - top) / HEIGHT));

// Start off z at (0, 0).
complex<double> z(0.0, 0.0);
\end{verbatim}
\caption{Some code.}
\label{fig:somecode}
\end{figure}

\Blindtext

\section{Research Question}

\Blindtext[1]

\section{Objectives}

\blinditemize

\chapter{Literature Review}

\Blindtext

\chapter{Methodology}

\blindmathpaper

\chapter{Results}

\Blindtext

\chapter{Discussion}

\Blindtext

\chapter{Conclusion and Future Work}

\section{Conclusion}

\Blindtext

\section{Future Work}

\Blindtext

\appendix

\chapter{Table of Tables of Tables}

\Blindtext[2]

\chapter{Survey Responses}

\Blindtext[7]

\printbibliography

\end{document}
